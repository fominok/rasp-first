\documentclass[12pt, a4paper]{article}
\usepackage[utf8]{inputenc}
\usepackage[T2A]{fontenc}
\usepackage[english, russian]{babel}
\usepackage[usenames,dvipsnames]{xcolor}
\usepackage{listings,a4wide,longtable,amsmath,amsfonts,graphicx,tikz}
\usepackage{indentfirst,verbatim}

\usepackage{minted}
\setminted[clojure]{
  fontsize=\scriptsize,
  baselinestretch=1.2,
  linenos,
  frame=lines
}

\begin{document}
\thispagestyle{empty}
\begin{center}
  {\large
    Университет ИТМО \\
    Кафедра Информатики и прикладной математики \\
  }
\end{center}
\vspace{\stretch{2}}
\begin{center}
  {\large
    Распределенные системы\\
  }
    \vspace{\stretch{1}}
  {\large
    Лабораторная работа 1\\
    ``Разработка и реализация консервативного алгоритма синхронизации''\\
  }

\end{center}
\vspace{\stretch{6}}
\begin{flushright}
  Работу выполнили студенты групп P4117 и P4115\\
  {\it Фомин Евгений\\
    Фролов Сергей \\
  }
\end{flushright}
\vspace{\stretch{4}}
\begin{center}
  2018
\end{center}
\newpage

\section{Исходные данные}
Разработать консервативный алгоритм синхронизации с нулевыми сообщениями
для имитации взаимодействия покупателя, банка и магазина.

\subsection{Краткие теоретические сведения}
Отличительной особенностью консервативного алгоритма синхронизации
с нулевыми сообщениями является
предотвращение временных парадоксов вместо работы над их
последствиями. Дополнительные проверки сообщений, которые
потенциально не нарушают хронологию, ведут к снижению
производтельности, но, в сравнении с оптимистическим алгоритмом,
это проще в реализации.

Так же для избежания тупиков необходимо продвигать модельное
время с введением нулевых сообщений. Для более эффективного их
использования каждый узел, зная минимальное время, затрачиваемое
им на выполнение задачи, может продвигать свое локальное время на это число,
так как по его выходным каналам сообщение с меньшей временной
меткой при получении нового события отправить технически невозможно.

Таким образом, сообщения по линиям связи посылаются с неубывающими
временными метками, при отправке задействованы все линии связи,
будь то событие или нулевое сообщение, а если событий не произошло,
то локальное время узла продвигается на минимальное время выполнения
задачи, далее отправляются сообщения по завершенным событиям (время
которых не превосходит текущее локальное, внутренняя очередь) или
нулевые сообщения с новым локальным временем, если таковых нет.

\subsection{Описание проблемной области}
В качестве примера рассматриваются взаимоотношения между
банком, одним из его клиентов и магазином, где покупатель
совершает покупку на весь свой кредитный лимит.

Перед началом работы во внутреннюю очередь клиента установлены
задачи с указанием времени окончания (и рассылки сообщения об этом):
клиент кладет сумму на счет, идет в магазин и делает покупку
с помощью кредитной карты, отправляется в банкомат и не может
снять нужную сумму из-за нехватки средств, в фоне магазин отправляет
банку запрос на снятие денег. Для простоты все узлы продвигают
при необходимости модельное время на единицу.

\section{Код программы}
\inputminted{clojure}{core.clj}

Алгоритм реализован до 62 строки, далее -- предметная область
и тестовый пример. Функции, относящиеся к алгоритму:
\begin{description}
\item[get-lbts] Для указанных каналов получить сообщение
  с наименьшей временной меткой или \texttt{nil}, если
  хотя бы один из каналов пуст.
\item[extrude-local-queue]Получить безопасное сообщение
  из внутренней очереди и новое состояние, где это сообщение
  извлечено.
\item[extrude-from-chan!]Извлечь сообщение из канала и
  получить состояние с продвинутым локальным временем.
\item[generic-process]Общая логика работы любого из узлов:
  получение LBTS, извлечение из локальной очереди/из канала,
  выполнение и рассылка выполненных задач с нулевыми
  сообщениями.

\end{description}

\section{Вывод}
В ходе лабораторной работы была поставлена проблема
синхронизации времени в распределенных системах и реализован
один из алгоритмов, призванный ее решить. Решение было проверено
в многопоточной среде с икусственно замедленными узлами.

\end{document}
